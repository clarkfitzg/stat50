\documentclass[12pt, addpoints]{exam}

\printanswers
%\noprintanswers

% Change margin size
%\usepackage[margin=1in]{geometry}   
 
\usepackage{graphicx}

\begin{document}

\textbf{Quiz - 31 January 2020}
\hfill
\textbf{STAT 50 - Prof. Fitzgerald}

\vspace{10mm}
 
\makebox[\textwidth]{Name:\enspace\hrulefill}
%%%%%%%%%%%%%%%%%%%%%%%%%%%%%%%%%%%%%%%%%%%%%%%%%%%%%%%%%%%%

\vspace{5mm}

Answer all questions, front and back.
Show your work.

\begin{center}
\gradetable[v][questions]
\end{center}

\vspace{5mm}
\begin{questions}

\question[4]
\textbf{True or False}, the method described produces a simple random sample of a population of 25 students in a particular class.
 
\begin{parts}

\part
Arrange students alphabetically by last name, and then select the first 5 students.
\begin{solution} 
False, students with last names starting with ``A'' are more likely to be included in sample.
\end{solution}
\vspace{\stretch{1}}

\part
Arrange students alphabetically by last name, and then select the 5th, 10th, 15th, 20th and 25th students.
\begin{solution} 
False, students with last names starting with ``A'' are less likely to be included in sample.
\end{solution}
\vspace{\stretch{1}}

\part 
Arrange students in a random permutation, and then select the first 5 students.

\emph{Hint: A random permutation means to arrange the students in a random order, such that all $25!$ permutations of students are equally likely.}
\begin{solution} 
True, each student is equally likely to appear among the first 5 students in a random permutation.
\end{solution}
\vspace{\stretch{1}}

\part
Arrange students in a random permutation, and then select the 5th, 10th, 15th, 20th and 25th students.
\begin{solution} 
True, each student is equally likely to appear as the 5th, 10th, 15th, 20th, and 25th student in a random permutation.
\end{solution}
\vspace{\stretch{1}}

\end{parts}

%%%%%%%%%%%%%%%%%%%%%%%%%%%%%%%%%%%%%%%%%%%%%%%%%%%%%%%%%%%%
\vspace{\stretch{1}}
\newpage
\question Here is the histogram for the student loan data we saw in class.
Note the bars all have widths of 20,000.

\begin{center}
\includegraphics[height=100mm]{first_student_loan_freq.pdf}
\end{center}

\begin{parts}

\part[2]

Approximately how many loans are between 20,000 and 60,000 dollars?

\begin{solution} 
$2,200 \pm 200$ 

There are around 1,600 loans between 20,000 to 40,000 dollars, and around 600 loans between 40,000 and 60,000 dollars.
So around $1,600 + 600 = 2,200$ loans are between 20,000 and 60,000 dollars.
\end{solution}
\vspace{\stretch{1}}

\part[4]
Based solely on this histogram, what are upper and lower bounds for $\min(x)$, the minimum loan amount?

\emph{Write your answer in the form $a \leq \min(x) \leq b$, such that $b - a$ is as small as possible.}

\begin{solution} 
$0 \leq \min(x) \leq 20,000$

The minimum value must be in the box farthest to the left.
That box includes values from 0 to 20,000.
\end{solution}
\vspace{\stretch{1}}

\end{parts}


\end{questions}
\end{document}
